\problemname{Mumble Rap}
\noindent
Willó lyssnar på ett flertal olika musikgenrer, och han brukar snacka med båda sina vänner om hans favoritartister. 
En dag frågar Willó sina föräldrar ifall de skulle kunna betala för en biljett till en musikfestival. 
Föräldrarna gick först med på det, men Willós mamma blev tveksam när hon insåg att det inte var vilken musikfestival som helst, utan det var en mumble rap musikfestival!

\noindent
Willós mamma frågade Willó hur han kan ha intresse för en så oförståelig musikgenre, då hon tyckte det var svårt att urskilja tal från ord som artisten mumlar. 
Enligt Willós mamma är mumble rap känt för att vara en musikgenre där artisten alltid rappar om hur mycket pengar artisten äger på ett mumlande sätt. 
Men Willó ville visa kärlek för mumble rap och bevisa att hans mamma har fel, och har kommit på en suverän plan: 
Willó vill skriva ett program som kan identifiera det största talet ur en sträng av karaktärer som en artist mumlat. 

\section*{Indata}
\noindent
Första raden består av ett heltal $1 \leqslant N \leqslant 10^6 $, antalet karaktärer som befinner sig i strängen.
Andra raden består av $N$ karaktärer i en sträng utan mellanslag, där varje karaktär är antingen en bokstav från ``a''-``z'',``A''-``Z'', en symbol (``.,;:?!''), eller en siffra mellan $0$ och $9$, inklusive $0$ och $9$.

\section*{Utdata}
\noindent
Skriv ut ett heltal $0 \leqslant A \leqslant 10^9$ - Det största heltalet som nämns i strängen. Det är garanterat att det alltid finns ett tal i strängen. 

\section*{Poängsättning}
Din lösning kommer att testas på en mängd testfallsgrupper.
\noindent
För att få poäng för en grupp så måste du klara alla testfall i gruppen.

\noindent
\begin{tabular}{| l | l | l |}
\hline
  Grupp & Poängvärde & Gränser \\ \hline
  $1$    & $20$       &  $A \leqslant 9$ \\ \hline
  $2$    & $20$       &  Det största talet i strängen är det sista talet som uppstår i strängen. \\ \hline
  $3$    & $30$       &  $N \leqslant 10^3$ \\ \hline
  $4$    & $30$       &  Inga ytterligare begränsningar \\ \hline
\end{tabular}
