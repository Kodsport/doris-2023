\problemname{Sångbok}
\noindent
Doris ska vara toastmaster på en sittning med sin sektion på universitet. Under denna ska de så klart sjunga en massa sånger! 
Eftersom Doris har dålig koll på låtarna i deras sångbok har hon valt att istället välja så många låtar som möjligt under de $t$ minuter som är allokerade för sång. 
Eftersom hon är i övrigt upptagen med att välja vem som ska tala, och vad som får sägas så vill hon ha hjälp med detta.
\section*{Indata}
\noindent
Börjar med en rad med två heltal, antalet minuter allokerade för sång, $t < 10^5$, och antalet låtar i sångboken, $n< 10^5$. Sedan följer en rad med $n$ heltal, antalet sekunder hon uppskattar att det tar att sjunga varje sång $s_i < t \cdot 60$.
\section*{Utdata}
\noindent
En rad med ett heltal, antalet sekunder det minimalt tar att sjunga det största möjliga antalet sånger.
\section*{Poängsättning}
\noindent
Din lösning kommer att testas på en mängd testfallsgrupper.
För att få poäng för en grupp så måste du klara alla testfall i gruppen.

\noindent
\begin{tabular}{| l | l | l |}
\hline
  Grupp & Poängvärde & Gränser \\ \hline
  $1$    & $20$       &  Alla låtar är tar lika lång tid att sjunga \\ \hline 
  $2$    & $40$       &  $n \leq 16$ \\ \hline
  $3$    & $40$       &  Inga ytterligare begränsningar \\ \hline
\end{tabular}
