\problemname{Rationalisering}
\noindent
Efter att ha läst en bok om Pythagoras har Doris blivit övertygad om att det inte finns några irrationella tal!
Hon menar alltså att varje tal faktiskt kan skrivas som kvoten av två heltal $A$ och $B$.
I denna anda har hon bestämt sig för att hitta dessa kvoter för några positiva fysiska konstanter.
Eftersom det finns en viss felmarginal $F$ i alla konstanter nöjer hon sig dock med att hitta en kvot som är mellan $C-F$ och $C+F$.
Dessutom tycker hon inte om onödigt långa tal, så hon vill hitta det paret $A$, $B$, vars kvot ligger inom denna felmarginal och har minst $A$. Om det finns flera sådana $A$ väljer hon det paret $A$, $B$ med minst $B$.
Kan du hjälpa henne med detta?

\section*{Indata}
\noindent
Första raden innehåller två heltal $C, F$, ($0 \leq C \leq 10$), ($10^{-8} \leq F \leq 0.1$), det önskade talet och den felmarginal som tillåts, båda på decimalform med maximalt 6 värdesiffror.

\section*{Utdata}
\noindent
Skriv ut två rader som vardera innehåller ett positivt heltal, $A$ respektive $B$. Det är garanterat att dessa är mindre än $10^6$.

\section*{Poängsättning}
\noindent
Din lösning kommer att testas på en mängd testfallsgrupper.
För att få poäng för en grupp så måste du klara alla testfall i gruppen.

\noindent
\begin{tabular}{| l | l | l |}
\hline
  Grupp & Poängvärde & Gränser \\ \hline
  $1$    & $20$       &  $B = 1000$ \\ \hline 
  $2$    & $40$       &  $A, B \leq 10^{3}$ \\ \hline
  $3$    & $40$       &  Inga ytterligare begränsningar \\ \hline
\end{tabular}
