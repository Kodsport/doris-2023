\problemname{Rationalization}
\noindent
Having recently read about Pythagoras, Doris has become convinced that there are no irrational numbers!
She claims that every number can be written as the quotient of two whole numbers $A$ and $B$.
As an exercise, she has decided to find these ratios for some positive physical constants.
Because every constant has some error $F$, she only wants a ratio with a quotient between $C-F$ and $C+F$.
Furthermore, since she dislikes long numbers, she wants to find the pair $A$, $B$, whose quotient lies within the interval and has the smallest $A$. If there are many such $A$, she chooses the pair $A$, $B$ with smaller $B$.
Can you help her accomplish this goal?

\section*{Input}
\noindent
The first contains two positive real numbers $C, F$, ($0 \leq C \leq 10$), ($10^{-8} \leq F \leq 0.1$), the desired quotient and the margin for error, both in decimal form with at most 6 significant digits.

\section*{Output}
\noindent
Print two lines with one positive integer each, $A$ and $B$. It is guaranteed that these are less than $10^6$.

\section*{Grading}
\noindent
Your solution will be tested on a number of test-case groups.
To receive points for a group, your solution must correctly solve every test-case in the group.

\noindent
\begin{tabular}{| l | l | l |}
\hline
  Group & Point value & Restrictions \\ \hline
  $1$    & $20$       &  $B = 1000$ \\ \hline 
  $2$    & $40$       &  $A, B \leq 10^{3}$ \\ \hline
  $3$    & $40$       &  No further restrictions \\ \hline
\end{tabular}
